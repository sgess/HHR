%% ****** Start of file apstemplate.tex ****** %
%%
%%
%%   This file is part of the APS files in the REVTeX 4 distribution.
%%   Version 4.1r of REVTeX, August 2010
%%
%%
%%   Copyright (c) 2001, 2009, 2010 The American Physical Society.
%%
%%   See the REVTeX 4 README file for restrictions and more information.
%%
%
% This is a template for producing manuscripts for use with REVTEX 4.0
% Copy this file to another name and then work on that file.
% That way, you always have this original template file to use.
%
% Group addresses by affiliation; use superscriptaddress for long
% author lists, or if there are many overlapping affiliations.
% For Phys. Rev. appearance, change preprint to twocolumn.
% Choose pra, prb, prc, prd, pre, prl, prstab, prstper, or rmp for journal
%  Add 'draft' option to mark overfull boxes with black boxes
%  Add 'showpacs' option to make PACS codes appear
%  Add 'showkeys' option to make keywords appear
\documentclass[aps,prl,preprint,groupedaddress]{revtex4-1}
%\documentclass[aps,prl,preprint,superscriptaddress]{revtex4-1}
%\documentclass[aps,prl,reprint,groupedaddress]{revtex4-1}

% You should use BibTeX and apsrev.bst for references
% Choosing a journal automatically selects the correct APS
% BibTeX style file (bst file), so only uncomment the line
% below if necessary.
%\bibliographystyle{apsrev4-1}

\begin{document}

% Use the \preprint command to place your local institutional report
% number in the upper righthand corner of the title page in preprint mode.
% Multiple \preprint commands are allowed.
% Use the 'preprintnumbers' class option to override journal defaults
% to display numbers if necessary
%\preprint{}

%Title of paper
\title{Multibunch Beam Physics at FACET}

% repeat the \author .. \affiliation  etc. as needed
% \email, \thanks, \homepage, \altaffiliation all apply to the current
% author. Explanatory text should go in the []'s, actual e-mail
% address or url should go in the {}'s for \email and \homepage.
% Please use the appropriate macro foreach each type of information

% \affiliation command applies to all authors since the last
% \affiliation command. The \affiliation command should follow the
% other information
% \affiliation can be followed by \email, \homepage, \thanks as well.
\author{Spencer Gessner}
\email[]{sgess@slac.stanford.edu}

\author{Karl Bane}
\author{Franz-Joseph Decker}
\author{Keith Jobe}
\author{Marc Ross}
\affiliation{SLAC}


\date{\today}

\begin{abstract}
We propose to operate the FACET linac with multiple electron bunches per RF pulse. We will produce two electron bunches at the thermionic cathode, separated by integer multiples of 5.6 ns, the period of the subharmonic buncher. Several accelerator subsystems will be significantly affected by the transition from one to two electron bunches per RF pulse. These include the injector, North Damping Ring (NDR), North Ring to Linac (NRTL), and the main linac. In addition, the primary beam diagnostics, including BPMs, will not be able to resolve the individual bunches at their closest separations. We discuss methods to extract information from the diagnostics in this case as well as the use of Lucretia for multibunch beamline simulations.
\end{abstract}

% insert suggested PACS numbers in braces on next line
\pacs{}
% insert suggested keywords - APS authors don't need to do this
%\keywords{}

%\maketitle must follow title, authors, abstract, \pacs, and \keywords
\maketitle

% body of paper here - Use proper section commands
% References should be done using the \cite, \ref, and \label commands
\section{Intorduction}
% Put \label in argument of \section for cross-referencing
%\section{\label{}}
Plasma wakefield studies are normally conducted as single-shot experiments. Here, a single-shot means that the plasma returns to its original state before the subsequent bunch passes through the plasma. The time scale for the plasma to return to equilibrium is 10-100 ns, which is comparable to the bunch separation in proposed linear colliders. It is possible to perform non-equilibrium plasma wakefield acceleration (PWFA) at FACET by configuring the linac for multibunch operation. In this paper, we discuss changes in machine operation for multibunch studies and the relevant accelerator physics.
\section{Injector}
The electron source is a thermionic cathode driven by a multi-channel pulser. The pulser is composed of two planar triode amplifiers that couple into a common amplifier that drives the cathode. The two triodes can be fired independently to produce electron pulses from the cathode with arbitrary separation. The maximum charge per pulse is roughly 27 nC, which is almost an order of magnitude greater than the 3.2 nC used for PWFA~\cite{gun}.

Charge from the gun is captured by the subharmonic buncher in buckets separated by integer multiples of 5.6 ns. The subharmonic buncher runs at 178.5 MHz, or $1/16$ the S-band frequency. Therefore, the smallest feasible bunch separation with good charge capture is 5.6 ns, or 16 S-band buckets. The subharmonic buncher is followed by a 10 cm S-band buncher in Sector 0 and on-crest acceleration in Sector 1, reaching an energy of 1.19 GeV before being diverted to the North Damping Ring. 

In Sectors 0 and 1, the beam energy is low and the bunch is strongly affected by space charge forces and wakefields. In addition, there will be beam loading in the buncher cavities resulting in a phase shift for the trailing bunch, and beam loading in the accelerating cavities that reduces the energy gain of the trailing bunch. It is unnecessary to calculate the effect of these actions on the electron bunch if it is possible to get full charge capture in the ring. The bunch phase space at the exit of the ring reflects the equilibrium emittance of the damping ring rather than the phase space of the bunch upon entering the ring. We will assume that it is possible to get full charge capture for two 3.2 nC bunches in the ring until proven otherwise.

\section{North Damping Ring}
The North Damping Ring (NDR) reduces the beam emittance $\epsilon$ through the synchrotron radiation damping mechanism. Low emittance beams are needed for PWFA experiments because the final spot size is proportional to $\epsilon^{1/2}$. The electron bunches are diverted from the linac to the NDR by a dipole magnet at the Damping Ring Interaction Point (DRIP). The bunches traverse the North Ring to Linac (NRTL) arc and enters the NDR via the injection septum magnet and injection kicker.  

\subsection{Injection Kicker}
The injection kicker is a pulsed magnet with a pulse length less than 60 ns that provides a 7 mead kick to the beam~\cite{kicker}. The kicker is controlled by a thyratron that produces a 40 kV pulse with an extremely fast rise and fall time. This magnet represents the first potential constraint on bunch spacing in the FACET linac. If the two electron bunches are significantly close to each other (less than 16.8 ns apart) they should both receive a strong enough kick to be captured in the ring. It does not matter if the two bunches see a different field in the magnet, so long as they see a strong enough deflection to be captured in the ring. Any orbital errors will be damped. For longer separations (say 33.6 ns), the two bunches will see roughly half of the peak field in the injector and will not be captured in the ring. For the longest separation (61.6 ns), it is possible to inject the beams on separate pulses. However, we will not satisfy the tolerance on the extraction kicker field for the longest bunch separation given the current hardware configuration.

\subsection{Single Bunch Instability}
The single bunch instability in the NDR is a microwave instability driven by inductive impedances in the ring vacuum chamber. When the beam enters the ring it is relatively long, but synchrotron radiation damping cause the beam to shrink down to a minimum bunch length determined by the shape of the RF potential in the ring. If there were no ring instabilities, a 3.2 nC bunch would damp to an equilibrium bunch length of about 6 mm. However, as the beam reduces in size, the peak beam current increases. Larger peak currents excite high frequency impedances with increasing strength, eventually exciting a very high frequency resonance. The high frequency resonance corresponds to a short wavelength wakefield in the time domain. The wakefield modulates the bunch energy spectrum on length scales shorter than the bunch length, and the bunch rapidly lengthens as particles with different energies follow different paths around the ring. The timescale for the 

Since 1994, the threshold for the instability has been roughly $1.5 \times 10^{10}$ electrons per bunch~\cite{sawtooth}. At $2 \times 10^{10}$ electrons per bunch, we will exceed the threshold. However, the effect of the instability is small at this bunch charge and the machine is usually run in this mode. Using two electron bunches in the ring instead of one should have no impact on the strength of the instability or the threshold, because the instability is a single bunch, short wavelength effect. Nevertheless, this effect is worth studying. We will use the existing synchrotron light optical system to measure the bunch profile using a streak camera.

\subsection{$\pi$-Mode Instability}
The $\pi$-mode instability is a coupling that can occur between the two electron bunches in the ring. A single bunch executes a synchrotron oscillation with respect to the fundamental frequency of the RF cavities as it circles the ring. A second bunch in the ring will also execute a synchrotron oscillation in the RF cavities. Now we consider the problem of coupled oscillators. The two bunches will oscillate with respect to one another and the fundamental RF frequency. The coupled motion is a superposition of two normal mode oscillations. The 0-mode is a net oscillation  A $\pi$-mode instability arrises when the two bunches 

\subsection{Extraction Kicker}
Discuss limits on extraction pulse/

\section{North Ring to Linac}
Discuss beam loading induced phase shift and jitter.

\section{Linac}
Discuss general linac orbit considerations and diagnostics limitations.

\subsection{Beam Loading}
Describe energy spread compensation with SLED timing. Phase shift in Sectors 2-10.

\subsection{Transverse Wakefields}
Discuss dipole modes and flat top approximation.

\section{Interaction Point}
Discuss effect of energy spread at final focus. Use of ICCD for beam diagnostics.

\section{Lucretia}
Do Lucretia sims.

\section{Conclusion}
Finito.

% If in two-column mode, this environment will change to single-column
% format so that long equations can be displayed. Use
% sparingly.
%\begin{widetext}
% put long equation here
%\end{widetext}

% figures should be put into the text as floats.
% Use the graphics or graphicx packages (distributed with LaTeX2e)
% and the \includegraphics macro defined in those packages.
% See the LaTeX Graphics Companion by Michel Goosens, Sebastian Rahtz,
% and Frank Mittelbach for instance.
%
% Here is an example of the general form of a figure:
% Fill in the caption in the braces of the \caption{} command. Put the label
% that you will use with \ref{} command in the braces of the \label{} command.
% Use the figure* environment if the figure should span across the
% entire page. There is no need to do explicit centering.

% \begin{figure}
% \includegraphics{}%
% \caption{\label{}}
% \end{figure}

% Surround figure environment with turnpage environment for landscape
% figure
% \begin{turnpage}
% \begin{figure}
% \includegraphics{}%
% \caption{\label{}}
% \end{figure}
% \end{turnpage}

% tables should appear as floats within the text
%
% Here is an example of the general form of a table:
% Fill in the caption in the braces of the \caption{} command. Put the label
% that you will use with \ref{} command in the braces of the \label{} command.
% Insert the column specifiers (l, r, c, d, etc.) in the empty braces of the
% \begin{tabular}{} command.
% The ruledtabular enviroment adds doubled rules to table and sets a
% reasonable default table settings.
% Use the table* environment to get a full-width table in two-column
% Add \usepackage{longtable} and the longtable (or longtable*}
% environment for nicely formatted long tables. Or use the the [H]
% placement option to break a long table (with less control than 
% in longtable).
% \begin{table}%[H] add [H] placement to break table across pages
% \caption{\label{}}
% \begin{ruledtabular}
% \begin{tabular}{}
% Lines of table here ending with \\
% \end{tabular}
% \end{ruledtabular}
% \end{table}

% Surround table environment with turnpage environment for landscape
% table
% \begin{turnpage}
% \begin{table}
% \caption{\label{}}
% \begin{ruledtabular}
% \begin{tabular}{}
% \end{tabular}
% \end{ruledtabular}
% \end{table}
% \end{turnpage}

% Specify following sections are appendices. Use \appendix* if there
% only one appendix.
%\appendix
%\section{}

% If you have acknowledgments, this puts in the proper section head.
\begin{acknowledgments}
Thanks everybody.
\end{acknowledgments}

% Create the reference section using BibTeX:
%\bibliography{high_rep_rateNotes}{}
\begin{thebibliography}{9}
% 
\bibitem{gun} 
  M.~J.~Browne, J.~E.~Clendenin, P.~L.~Corredoura, R.~K.~Jobe, R.~F.~Koontz, and J.~Sodja, 
  ``A Multi-Channel Pulser for the SLC Thermionic Electron Source", 
  Particle Accelerator Conference, Vancouver, 1985. 
  SLAC-PUB 3546.
  
\bibitem{kicker}
  T. Mattison, R. Cassel, A. Donaldson, D. Gough, G. Gross, A. Harvey, D. Hutchinson, and M. Nguyen,
  ``Status of SLC Damping Ring Kicker System",
  Particle Accelerator Conference, San Francisco, 1991.
  SLAC-PUB 5462
  
\bibitem{pulse}
  T. Mattison, R. Cassel, A. Donaldson, H. Fischer, and D. Gough,
  ``Pulse Shape Adjustment for the SLC Damping Ring Kickers",
  Particle Accelerator Conference, San Francisco, 1991.
  SLAC-PUB 5461.
  
\bibitem{sawtooth}
  K. Bane, J. Bowers, A. Chao, T. Chen, F. J. Decker, R. L. Holtzapple, P. Krejcik, T. Limberg, A. Lisin, B. McKee, M. G. Minty, C.-K. Ng, M. Pietryka, B. Podobedov, A. Rackelmann, C. Rago, T. Raubenheimer, M. C. Ross, R. H. Siemann, C. Simopoulos, W. Spence, J. Spencer, R. Stege, F. Tian, J. Turner, J. Weinberg, D. Whittum, D. Wright, F. Zimmermann,
  ``High-Intensity Single Bunch Instability Behavior In The New SLC Damping Ring Vacuum Chamber",
  Particle Accelerator Conference, Dallas, 1995.
  SLAC-PUB 6894.
    
\end{thebibliography}

\end{document}
%
% ****** End of file apstemplate.tex ******wne

